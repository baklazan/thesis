\chapter{Ciele práce}

\label{kap:ciele}

Pri niektorých využitiach DNA sekvenovania sa sekvenuje vzorka, o ktorej je známe, že by
sa mala podobať na inú, už osekvenovanú DNA. 
Cieľom sekvenovania je potom zistiť, ako sa tieto dve DNA postupnosti líšia. Jedným z takýchto využití 
je napríklad zisťovanie rezistencie baktérií na antibiotiká \cite{Bradley2015}.

V našej práci sa budeme zaoberať nasledujúcim scenárom. Máme nejakú známu postupnosť
dusíkatých báz, ktorú budeme nazývať \emph{referencia}. Ďalej máme vzorku DNA, o ktorej vieme,
že sa od referencie líši len veľmi málo. Táto vzorka bola spracovaná MinIONom, máme teda k dispozícii
nameraný surový signál z jednotlivých čítaní. Naším cieľom je identifikovať varianty v sekvenovanej
vzorke vzhľadom na referenciu. Ideálne by bolo vedieť s dobrou presnosťou určovať varianty už z jedného čítania.

Snažíme sa teda navrhnúť algoritmus s nasledovným vstupom a výstupom (neformálne):

\begin{description}

\item[Vstup:] \begin{description}
\item referenčná postupnosť dusíkatých báz,
\item postupnosť nameraných hodnôt surového signálu,
\item \textit{nepovinné}: odhad očakávaného množstva variantov
\end{description}

\item[Výstup:] \begin{description}
\item popis nájdených variantov (pozícia, typ, skóre)
\end{description}
\end{description}

Náš algoritmus bude nevyhnutne robiť chyby. V niektorých prípadoch nezvládne nájsť variant, ktorý vzorka
obsahovala (falošné odmietnutie), v iných prípadoch nájde variant, ktorý neexistuje (falošné prijatie). 
V niektorých aplikáciách môže byť cena za chyby jedného druhu väčšia, než cena za chyby opačného 
druhu. Algoritmus preto vráti ku každému z nájdených variantov aj skóre, indikujúce istotu algoritmu, že 
naozaj ide o variant. Znižovaním minimálneho skóre, ktoré budeme vyžadovať, aby sme nájdený variant 
považovali za skutočný, bude možné znížit množstvo falošných odmietnutí za cenu zvýšenia množstva 
falošných prijatí, a obrátene.

\section{Rozdiel oproti určovaniu báz}

Jedným z možných riešení nášho problému je určiť zo signálu bázy pomocou niektorého z existujúcich 
basecallerov a následne už len zisťovať odlišnosti dvoch postupností báz. Problémom tohoto riešenia
je nízka presnosť (ak nemáme veľa prekrývajúcich sa čítaní). 

Pri tomto prístupe však basecaller vôbec
nevyužíva fakt, že sekvenovaná postupnosť sa podobá na referenciu. Uvažuje teda podstatne väčší 
priestor možných výsledných sekvencií, než je nutné. V dôsledku toho upredňostňuje sekvencie, ktoré 
lepšie vysvetľujú pozorovaný signál, aj keď môžu byť výrazne vzdialené od referencie.
Zmenšenie priestoru uvažovaných sekvencií môže navyše znížiť výpočtovú náročnosť určovania báz, prípadne
umožniť použitie presnejších techník, ktoré by za normálnych okolností boli príliš časovo náročné.
