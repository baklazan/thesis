\chapter*{Záver}  % chapter* je necislovana kapitola
\addcontentsline{toc}{chapter}{Záver} % rucne pridanie do obsahu
\markboth{Záver}{Záver} % vyriesenie hlaviciek

Cieľom tejto práce bolo navrhnúť nový algoritmus na identifikáciu variantov zo surových dát sekvenačného
prístroja MinION, ktorý by bol presnejší, než porovnanie
referencie s výstupom z prekladača báz. Toto spresnenie sme sa snažili dosiahnuť využitím informácie,
ktorú prekladač báz nevyužíva: že sekvenovaná postupnosť sa musí podobať na referenciu.

Vytvorili a implementovali sme pravdepodobnostný model, ktorý identifikuje SNPy na základe signálu z jedného čítania.
V ideálnych podmienkach, teda keď sekvenovaná postupnosť neobsahuje žiadne varianty okrem malého množstva SNPov a dopredu vieme, koľko SNPov máme očakávať, je náš model pri testovaní presnejší,
než porovnanie výstupu z prekladača báz s referenciou.  Hoci k presnej identifikácii variantov všetkých druhov
máme ešte ďaleko, tento výsledok vzbudzuje nádej, že využitie znalosti referencie pri spracovaní signálu
môže naozaj spresniť identifikáciu variantov.

Pravdepodobnostný model, ktorý sme navrhli, by sa dal v budúcnosti rozšíriť dvoma spôsobmi.
Môžeme ho rozšíriť na model, ktorý identifikuje SNPy aj z viacerých čítaní jednej sekvencie.
Pre niektoré hypotézy dostaneme podmienené pravdepodobnosti z viacerých čítaní. Aposteriórnu
pravdepodobnosť danej hypotézy budeme potom počítať z apriórnej pravdepodobnosti a všetkých týchto
podmienených pravdepodobností.

Ďalšou možnosťou je rozšíriť model, aby bolo pomocou neho možné identifikovať aj krátke inzercie
a delécie. Stačí pre jednotlivé pozície začať uvažovať aj hypotézy hovoriace, že na danej pozícii je inzercia,
prípadne delécia. Rozšírenie na dlhšie inzercie je problematické, lebo s dĺžkou inzercie exponenciálne rastie
počet možností, aké bázy boli do postupnosti vložené a tým aj počet hypotéz.

Presnejšie, než rozšírenia nášho modelu, však možno budú fungovať úplne iné prístupy.
Najlepšie súčasné prekladače báz pre prístroj MinION sú založené na neurónových sieťach, je teda dosť dobre možné,
že aj identifikácia variantov sa bude dať riešiť presnejšie pomocou neurónovej siete.