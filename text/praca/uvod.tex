\chapter*{Úvod} % chapter* je necislovana kapitola
\addcontentsline{toc}{chapter}{Úvod} % rucne pridanie do obsahu
\markboth{Úvod}{Úvod} % vyriesenie hlaviciek

DNA nesie genetickú informáciu zakódovanú v postupnosti štyroch
dusíkatých báz: adenínu, cytozínu, guanínu a tymínu.
Od sedemdesiatych rokov dvadsiateho storočia vznikajú stále nové
techniky ako sekvenovať DNA, teda ako pre fyzickú vzorku DNA zistiť
jej postupnosť báz.

Nanopórové sekvenovanie je technika sekvenovania DNA, pri ktorej jedno vlákno
DNA prechádza veľmi malým otvorom (nanopórom) a ovplyvňuje pritom elektrický
prúd prechádzajúci týmto otvorom. Tento prúd sa meria a na základe jeho
zmien sa potom určuje, aké bázy prechádzali nanopórom. Nanopórové sekvenovanie
má oproti iným technikám sekvenovania niekoľko výhod,
umožňuje napríklad pomerne lacno a rýchlo sekvenovať DNA už v malých množstvách.
Jeho veľkou nevýhodou je však veľké množstvo chýb, ktorých sa pri sekvenovaní dopúšťa.

V niektorých aplikáciách sekvenujeme DNA, o ktorej dopredu vieme, že vyzerá, až na 
drobné odlišnosti, ako nejaká známa postupnosť báz.
Cieľom sekvenovania vtedy býva práve nájdenie týchto odlišností (nazývaných aj varianty). Pri použití
nanopórového sekvenovania na hľadanie variantov narazíme na problém, že veľa
variantov, ktoré nájdeme, sú v skutočnosti iba nepresnosti v sekvenovaní. V tejto
práci sa snažíme nájsť spôsob, ako využiť dotatočnú informáciu o sekvenovanej DNA
(že sa podobá na známu postupnosť) už vo fáze prevádzania nameraného elektrického
prúdu na postupnosť báz a zvýšiť tým presnosť identifikácie variantov.

V kapitole \ref{kap:sekvenovanie} detailnejšie popíšeme proces sekvenovania DNA konkrétnym 
nanopórovým sekvenátorom MinION. V kapitole \ref{kap:ciele} presnejšie zadefinujeme cieľ našej
práce.
V kapitole \ref{kap:identifikacia_SNP} navrhneme spôsob, ako identifikovať najjednoduchší
druh variantov -- jednonukleotidové polymorfizmy.
V kapitole \ref{kap:testovanie} tento spôsob otestujeme na reálnych dátach z prístroja MinION.
 Najprv experimentálne nájdeme vhodné parametre pre náš model a potom
ho porovnáme s priamočiarejším prístupom, ktorý danú DNA osekvenuje a výsledok sekvenovania porovná
s postupnosťou, na ktorú sa má podobať.