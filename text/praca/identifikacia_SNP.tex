\chapter{Identifikácia jednonukleotidových polymorfizmov}

\label{kap:identifikacia_SNP} 

V tejto časti navrhneme techniku na odhaľovanie jednonukleotidových
polymorfizmov (SNP) v sekvenovaných dátach. Uvažujeme pritom najjednoduchší
možný scenár, keď sekvenovaná vzorka neobsahuje žiadne iné varianty a jednotlivé
SNP nie sú príliš blízko seba. Predpokladáme teda, že ak sa sekvenovaná vzorka 
líši od referencie v nejakej báze, v okolitých bázach sa tieto dve sekvencie
zhodujú.

\section{Pravdepodobnostný model}

Náš postup pri identifikácii SNP je založený na jednoduchom pravdepodobnostnom modeli.
Tento model odhaduje podmienené pravdepodobnosti typu 
,,Aká je pravdepodobnosť, že by postupnosť báz $D$ pri sekvenovaní vygenerovala signál $S$?''

Model predpokladá, že bázy sekvenovaného vlákna DNA postupne prechádzajú nanopórom, pričom
každá báza sa v nanopóre nachádza počas minimálne $M$ po sebe idúcich meraní signálu. Ku každej
nameranej hodnote signálu teda môžeme priradiť bázu, ktoré sa v danom čase nachádzala v nanopóre.
Tomuto priradeniu budeme hovoriť \emph{zarovnanie}.

\begin{definicia}

Nech $n, m, M \in \mathbb{Z}^+$, pričom $m \geq n M$. \emph{Zarovnaním $n$ hodnôt ku $m$ s minimálnou dĺžkou udalosti $M$} nazývame ľubovoľné zobrazenie $f: \{0, 1, \dots, m-1\} \rightarrow \{0, 1, \dots, n-1\}$
také, že:

\begin{itemize}
\item $f$ je neklesajúce, teda $\forall x, y \in \{0, 1, \dots, m-1\}: x < y \Rightarrow f(x) \leq f(y)$.
\item $\forall i \in \{0, 1, \dots, n-1\}: \abs*{\{x ~|~ f(x) = i\}} \geq M$.
\end{itemize}

Množinu všetkých zarovnaní $n$ hodnôt ku $m$ s minimálnou dĺžkou udalosti $M$ budeme označovať ako
$A(n, m, M)$. Táto množina je zrejme konečná.

\end{definicia}

Náš model ďalej predpokladá, že nameraný signál závisí iba od bázy, ktorá je práve v nanopóre, $c_1$
báz pred ňou a $c_2$ báz za ňou. To sa nedá aplikovať na prvých $c_1$ a posledných $c_2$ báz 
sekvenovanej postupnosti. Pri modelovaní preto budeme pre jednoduchosť predpokladať, že DNA postupnosť, 
ktorej signál modelujeme, išla cez nanopór ako súčasť nejakej dlhšej sekvencie. Na modelovanie
signálu, ktorý bol pri tejto postupnosti zaznamenaný, budeme teda potrebovať poznať aj 
kontext $c_1$ báz pred začiatkom postupnosti a $c_2$ báz za jej koncom.

\begin{definicia}

\emph{DNA sekvenciou dĺžky $n$ s kontextom $c_1$ a $c_2$} nazývame ľubovoľné zobrazenie 
$D: \{-c_1, -c_1+1, \dots, n+c_2 - 1\} \rightarrow \bazy$. Funkčnú hodnotu $D$ v danom bode $i$
budeme namiesto $D(i)$ značiť ako $D_i$.

\end{definicia}

Nech $k = c_1 + 1 + c_2$. Budeme hovoriť, že nejaký \kmer{} sa nachádza v nanopóre, ak báza, 
ktorá je v nanopóre, spolu s okolím $c_1$ báz pred ňou a $c_2$ báz za ňou tvorí tento \kmer{}.

Závislosť signálu od báz, ktoré sú v nanopóre a okolí je popísaná pomocou rozdelenia pravdepodobnosti.
Pre každý z $4^k$ možných \kmer{ov}  máme v našom modeli rozdelenie 
pravdepodobnosti pre hodnotu signálu, ktorý vznikne, ak sa tento \kmer{} nachádza v nanopóre.
Keďže pracujeme s digitalizovanou hodnotou signálu, existuje iba konečné množstvo hodnôt, ktoré
môže signál nadobudnúť. Rozdelenie pravdepodobnosti teda bude pravdepodobnostná funkcia.

\begin{definicia}

Množinu všetkých možných nameraných hodnôt signálu budeme označovať $\mathbb{S}$.

\end{definicia}

Čísla $M$, $c_1$, $c_2$ a pravdepodobnostné funkcie pre jednotlivé \kmer{y} sú jediné
parametre nášho modelu.

\begin{definicia}

TODO je štvorica $(M, c_1, c_2, \Delta)$, kde:

\begin{itemize}
\item $M \in \mathbb{Z}^+$.
\item $c_1, c_2 \in \mathbb{Z}_{\geq 0}$.
\item $\Delta$ je množina, ktorá pre každé $K \in \bazy^{c_1 + 1 + c_2}$ obsahuje funkciu
$\delta_K: \mathbb{S} \rightarrow \mathbb{R}$ takú, že $\sum\limits_{s \in \mathbb{S}} \delta_K(s) = 1$.
\end{itemize}
\end{definicia}

Pri odhadovaní pravdepodobnosti signálu budeme jednotlivé merania považovať za nezávislé udalosti.
Tento predpoklad je značne nerealistický, bol zvolený najmä pre svoju jednoduchosť.

\begin{definicia}

Nech $(M, c_1, c_2, \Delta)$ je TODO, $n, m \in \mathbb{Z}^+$, $D$ je DNA sekvencia dĺžky $n$ s konextom
$c_1$ a $c_2$, nech $S = (S_i)_{i=0}^{m-1}$ je postupnosť hodnôt z $\mathbb{S}$ a nech 
$a \in A(n, m, M)$. Pravdepodobnosť vygenerovania signálu $S$ pri prechode sekvencie $D$ nanopórom
pri zarovnaní $a$ podľa modelu $(M, c_1, c_2, \Delta)$ bude:

$$P(S ~|~ D, a) = \prod\limits_{i=0}^{m-1} \delta_{(D_{i-c_1}, \dots, D_{i+c_2-1})}(S_i) \text{.}$$

\end{definicia}

Na odhadnutie pravdepodobnosti vygenerovania signálu $S$ postupnosťou $D$ musíme uvažovať všetky možné
zarovnania z $A(n, m, M)$. Pre jednoduchosť budeme predpokladať, že všetky zarovnania z $A(n, m, M)$ sú rovnako pravdepodobné.

\begin{definicia}

Nech $(M, c_1, c_2, \Delta)$ je TODO, $n, m \in \mathbb{Z}^+$, $D$ je DNA sekvencia dĺžky $n$ s kontextom $c_1$ a $c_2$ a nech $S = (S_i)_{i=0}^{m-1}$ je postupnosť hodnôt z $\mathbb{S}$.
Pravdepodobnosť vygenerovania signálu $S$ pri sekvenovaní $D$ podľa modelu $(M, c_1, c_2, \Delta)$ je:

$$P(S ~|~ D) =  \frac{\sum\limits_{a \in A(n, m, M)} P(S ~|~ D, a)}{\abs*{A(n, m, M)}} \text{.}$$
\end{definicia}



\section{Niečo iné}
